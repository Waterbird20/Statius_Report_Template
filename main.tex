%% Template by Chloé Rouyer, Aimed for theses written at DIKU. 
%% This templaate is aimed to be easy to use and highly customizable. 
%% Contact if questions at chloerouyer.ml@gmail.com

%% Inspired from the template of Alex Jørgensen
%% https://www.overleaf.com/latex/templates/thesis-and-project-ucph/jnvpqdzrhrzr

%%%%% General Structure of the template %%%%%

\documentclass[12pt, oneside]{book} 
% The oneside option ensures that all pages are similar. the default option 'twosides' distinguishes between left and right pages.
\usepackage{KUstyle}
\usepackage[numbers, round]{natbib}



%% If you want to modify from the default margins
% \usepackage[left= 1.1in, right = 1.1in, top = 1.5in, bottom = 1.5in]{geometry} % Set the margins
% \usepackage{layouts} % Use in combination with the command %

%%%% Personal setup %%%%%

%% bibtex Important package to use full citations: bibentry
\usepackage{bibentry}
\nobibliography*

\bibliographystyle{plainnat}


%% You can put all packages and personalized commands in the following document.
% IMPORTANT! In order for the document to compile, one needs to use XeLaTeX or LuaLaTeX as compiler. This can be done in  Overleaf by Menu -> Settings -> Compiler -> Choose XeLaTeX/LuaLaTeX

%% These are the packages and commands that I used, feel free to remove and edit them.

%% Set the path for your images. Remember to edit all your includegrpahics to the correct location.
 \graphicspath{ {images/} }

%% Suggested Commands %%

%% Personal favourite command if you want to number a single line in a series of equations. Use an align* environement and use \numberthis specifically on the line that you want to number rather than using an align environement and \notag on every line you don't want to number.  
\newcommand\numberthis{\addtocounter{equation}{1}\tag{\theequation}}

%% If you have Dutch names in your references 
\DeclareRobustCommand{\VAN}[3]{#2} % proper Dutch 'van/de' capitalisation

%% If you want your theorems to include the Chapter numbers in the nmumbering
\usepackage{amsthm} %P ackage that gives the theorems environment

\newtheorem{prop}{Proposition}[chapter]
\newtheorem*{prop*}{Proposition}
\newtheorem{example}{Example}[chapter]
\newtheorem{proof_sketch}{Sketch of the Proof}[chapter]
\newtheorem{theorem}{Theorem}[chapter]
\newtheorem{lemma}{Lemma}[chapter]
\newtheorem*{lemma*}{Lemma}
\newtheorem{remark}{Remark}[chapter]
\newtheorem{definition}{Definition}[chapter]
\newtheorem{corollary}{Corollary}[chapter]
\newtheorem{notation}{Notation}[chapter]


%% Other packages I typically use. 

\usepackage{graphicx}
\usepackage{indentfirst}
\usepackage{multirow}
\usepackage{physics}
\usepackage{amsmath}
\usepackage{subfigure}
\usepackage{mathtools}
\usepackage{ragged2e}
\usepackage{esint}
\usepackage{wrapfig}
\usepackage{mdframed}
\usepackage{filecontents}

% \usepackage{geometry}
% \usepackage{setspace} % Package for linespacing
% \usepackage{tabularx} % Package for table
% \usepackage{fontspec}

% \usepackage[T1]{fontenc}    % use 8-bit T1 fonts
% \usepackage{url}            % simple URL typesetting
% \usepackage{booktabs}       % professional-quality tables
% \usepackage{amsfonts}       % blackboard math symbols
% \usepackage{nicefrac}       % compact symbols for 1/2, etc.


% \usepackage{xcolor}         % colors
% \usepackage{dsfont}
% \usepackage{bm}
% \usepackage{mathtools}
% \usepackage{cleveref}
% \usepackage{amsmath}
% \usepackage{amssymb}
% \usepackage{xifthen}
% \usepackage[ruled]{algorithm2e}
% \usepackage{algorithmic}


%%%% Template Related Setup %%%%%

%%% Front Page %%%
\ptype{Status Report}
\author{Donghun Jung}
\title{[Title]}
\subtitle{}
% \advisor{Advisor: [Advisor]}
\date{This report summarizes my studies over the past month until Feb 12 and is submitted on \today.}
%% This sentence is needed for PhD theses, but it can be replaced by the date for other projects.

\renewcommand{\contentsname}{Table of Contents}

%%% Report parameters %%%

%% Table of contents %%
% Set table of contents depth:  {1} only includes sections, {2} also includes subsections
 \setcounter{tocdepth}{1}

% Specific entries can be added manually with a command such as: 
% \addcontentsline{toc}{section}{[Title of Section]}  %Use section or chapter to decide how this extra element should be displayed
 
% remove the dots in the table of contents
%  \makeatletter
% \renewcommand\@dotsep{280}   % default value 4.5
% \makeatother

%% Section Numbering within the chapters %%
% number sections until subsubsections (set to 2 to number until subsections, and to 4 to number paragraphs)
 \setcounter{secnumdepth}{3}
 
%% Set Chapter title in header %% 
\usepackage{titlesec}
\usepackage{fancyhdr}

% plain style so that the header does not appear in the abstract
\fancypagestyle{plain}{         
\fancyhf{}
\fancyfoot[C]{\thepage}}%

% new style for the mainmatter with the line under the title
\renewcommand{\headrulewidth}{0pt}
\renewcommand{\chaptermark}[1]{\markboth{#1}{}}
\newcommand\mymainpagestyle{%
\fancyhf{}      
\fancyhead[L]{\nouppercase{\footnotesize{\chaptername~ \thechapter~ |~ \leftmark}} \renewcommand{\headrulewidth}{0.4pt} \headrule \renewcommand{\headrulewidth}{0pt}}
\setlength{\headheight}{25pt}
\fancyfoot[C]{\thepage}
}

%% alternate style for the mainmatter without the line under the title
% \renewcommand{\headrulewidth}{0pt}
% \renewcommand{\chaptermark}[1]{\markboth{#1}{}}
% \newcommand\mymainpagestyle{%
% \fancyhf{}      
% \fancyhead[L]{\nouppercase{\footnotesize{\chaptername~ \thechapter~ |~ \leftmark}} }
% \setlength{\headheight}{25pt}
% \fancyfoot[C]{\thepage}
% }

%% Note: if you want to use a different running title in the header, you can use the following structure when giving your title:
% \chapter{Long chapter title}
% \chaptermark{Short chapter title}


%%%%% Main %%%%%

\begin{document}

% \pagevalues  % If you are using the package layouts and want to get the current layout margin values so you can tune them. They are displayed on the first page if you activate this option. Remember to uncomment the \usepackage{layouts} at the beginning of this document. 


%%%% Introduction %%%%
\maketitle
\frontmatter % to get a different numbering of the frontmatter and mainmatter.
\pagestyle{plain} % not to get a header in the introduction pages

\newpage \ \newpage % to skip a page 

% The Abstract, Danish Abstract and Acknowledgements are not numbered here but still appear in the table of contents.
\section*{Summary}
\label{sec:summary}
\addcontentsline{toc}{section}{Summary} 
\input{chapters/Introduction/Summary.tex}

% \newpage
% \section*{Resumé}
% \label{sec:resume}
% \addcontentsline{toc}{section}{Resumé}
% [Danish version of the Abstract]

\newpage
\tableofcontents
\newpage

 
\newpage


%%%% Introduction %%%%
\mainmatter 
\mymainpagestyle{} % to get a header in the rest of the chapters (excpet the first page of the chapter)

% \chapter{Introduction}
% \label{chap:intro}
% [General introduction Chapter.]

% place your chapters there
\chapter{[Chapter 1]}
\label{chap:chap1}
% \begin{remark}
%  This work is published in the proceedings of COLT 2020. It was also presented as poster at the MLSS 2020 and DS3 2020 summer schools, where it won a best-poster award. 
%  \end{remark}

% The work presented in this chapter is based on a paper that has been published as: \bibentry{ExamplePub}. 

% Use \bibentry{} to get the reference details of the publication or \citet{} for a normal citation. 

% \newpage

\section*{Abstract}
[Abstract]
%% Place abstract here, and remove the \abstract commands

% \section{Introduction}
%% Place the rest of the boby of the paper here.

%% Remove the call to references if you just want them at the end of your thesis, as well as all potential extra submission details (for example checklists).




\section{[Section 1]}




% \section{Appendix}
%% The Appendix is now a section. Replace all the sections by subsections and the subsections by subsubsections [which exist because this thesis is in a book format].


% Conclusion Chapter
\chapter{What to do?}
\label{chap:conc}
\section*{What I've done and haven't}

\section*{What I'm going to do}


\bibentry{example}



% List of publications: The title appears as large as a Chapter header in the document, but as large as a section in the table of Contents, and there are no more page headers. 
\newpage
\phantomsection
\pagestyle{plain}
% \chapter*{List of Publications}
% \addcontentsline{toc}{section}{List of Publications}
% The work presented in this thesis has lead to the following publications. 


\begin{enumerate}
    \item \bibentry{ExamplePub}. 
    \item \bibentry{ExamplePub}.
    \item \bibentry{ExamplePub}.
\end{enumerate}

%% Here we use bibentry to get the full citation of the publications. 

\newpage
\addcontentsline{toc}{chapter}{References}
% \bibliographystyle{nature}
\bibliography{references}

\end{document}
